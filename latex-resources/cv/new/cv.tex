\documentclass{article}

\usepackage{titlesec}
\usepackage{titling}
\usepackage[margin=2.5cm]{geometry}
\usepackage{fontawesome}
\usepackage{paralist}

\setlength{\parindent}{0cm}

\titleformat{\section}
{\large\bf\uppercase}
{\hspace{-0.5cm}}
{0em}
{}[\hspace{-0.5cm} \titlerule]

\titleformat{\subsection}[runin]
{\bf}{}{0em}{}


\titlespacing{\subsection}
{0em}{0em}{0em}


\begin{document}

\title{Življenjepis}
\author{ŽIGA PERNE}

\renewcommand{\maketitle}{
\begin{center}
{\huge\bfseries
\theauthor}
\linebreak

\faGraduationCap \  Fakulteta za strojništvo UL\\
\faMapMarker\  Zoisova 10, 4000 Kranj\\
\faEnvelopeO \  ziga.perne@gmail.com \faPhone \  041 577 196\\
\faGithub \ github.com/zigaPerne

\vspace{1cm}


\end{center}
}

\maketitle

\section{O meni}
Sem Žiga Perne, magistrski študent na Fakulteti za strojništvo UL. Rojen sem bil 12. oktobra 1996 v Kranju, kjer sem obiskoval osnovno šolo in gimnazijo, na Fakulteto za strojništvo pa sem se vpisal leta 2017. Izbral sem visokošolski strokovni študijski program, saj sem se tako lahko že v drugem letniku usmeril v svoje področje zanimanja, in sicer v energetsko strojništvo.

Diplomiral sem v septembru leta 2020 pod mentorstvom doc. dr. Tineta Seljaka in v sodelovanju s podjetjem Syntech d.o.o. Tema zaključnega dela se glasi Analiza postopka uplinjanja v laboratorijski uplinjevalni napravi.

Po zaključenem dodiplomskem študiju sem se leta 2020 vpisal na magistrski študijski program, katerega obiskujem še danes. Trenutno sem študent prvega letnika smeri Energetska tehnika. 



\section{Izobrazba}

\subsection{Univerza v Ljubljani, Fakulteta za strojništvo}
\subsection{Dodiplomski študij:}

\hfill Projektno aplikativni program

\subsection{Magistrski študij:}

\hfill Energetska tehnika

\section{Kompetence}

\subsection{Programski jeziki:}
 
\hfill
Python, shell

\subsection{Programska oprema:}

\hfill
SolidWorks, Ansys Fluent, AVL Boost, Excel/Calc, LaTeX, git

\subsection{Operacijski sistemi:}

\hfill
Windows XP/7/10, GNU/Linux (Arch, Ubuntu)

\subsection{Jeziki:}

\hfill
Slovenščina, Angleščina

\section{Izku\v snje}

\subsection{Syntech d.o.o.} \hfill {\scriptsize poletje 2017, poletje 2018, šolsko leto 2019/20}
\subsection{}
{\em Sodelovanje pri razvoju uplinjevalne naprave}
\smallskip
\begin{compactitem}
\setlength{\itemindent}{-8.5mm}
\setlength{\listparindent}{-2cm}
	\item[$ \cdot  $] Nadzorovanje poteka testiranja
	\item[$ \cdot  $] Sodelovanje pri zasnovi sistema za doziranje odpadkov v uplinjevalno napravo 
	\item[$ \cdot  $] Splošno vzdrževanje in čiščenje naprave
	\item[$ \cdot  $] Prevod strokovnega in promocijskega materiala v Angleščino
	\item[$ \cdot  $] Raziskovalno delo: pregled stanja tehnologije uplinjanja, zasnova laboratorijske eksperimentalne 
	\item[]uplinjevalne naprave (sodelovanje s FS - diplomsko delo)
\end{compactitem}

\pagebreak

\subsection{Servis Kostanjevec d.o.o.} {\scriptsize \hfill poletje 2019}
\subsection{}
{\em Vzdrževanje strojev in naprav}
\smallskip
\begin{compactitem}
\setlength{\itemindent}{-8.5mm}
	\item[$ \cdot  $]Delo v delavnici - servis raznih strojev
	\item[$ \cdot  $]Redno vzdrževanje is servis v industrijskem obratu Danfoss Trata Kamnik
\end{compactitem}

\bigskip

\subsection{Laboratorij LICEM, FS} {\scriptsize \hfill marec/april 2020}
\subsection{}
{\em Praktično usposabljanje}
\smallskip
\begin{compactitem}
\setlength{\itemindent}{-8.5mm}
	\item[$ \cdot  $]Delo s programom AVL Boost
	\item[$ \cdot  $]{\em Design of Experiment} v MATLAB MBC Model Fitting 
\end{compactitem}


\end{document}
